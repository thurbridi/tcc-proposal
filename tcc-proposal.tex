\documentclass{ufsc-thesis}
\input{preamble.tex}

%----------------------------------------------------------------------
% Comandos criados pelo usuário
%----------------------------------------------------------------------
\newcommand{\todo}[1]{{\color{red}{#1}}}
\newcommand{\critical}[1]{{\color{red}\textbf{{#1}}}}
\newcommand{\verycritical}[1]{{\color{red}\textbf{\uppercase{{#1}}}}}

\definecolor{shadecolor}{rgb}{0.8,0.8,0.8}
\newcommand\VRule[1][\arrayrulewidth]{\vrule width #1}

\newcommand{\shadecell}{{\cellcolor{shadecolor}}}

\usepackage[a5paper,inner=2.5cm,outer=1.5cm,top=2.0cm,bottom=1.5cm,head=0.7cm,foot=0.7cm]{geometry}
\renewcommand{\normalsize}{\small}

%----------------------------------------------------------------------
% Identificadores do trabalho
% Usados para preencher os elementos pré-textuais
%----------------------------------------------------------------------
\titulo{Seleção automática de atributos sísmicos para classificação de fácies por redes de Kohonen}
\autor{Arthur Bridi Guazzelli}
\data{\today}
\instituicao{Universidade Federal de Santa Catarina}
\local{Florianópolis}
\tipotrabalho{Trabalho de Conclusão de Curso}
\orientador{Mauro Roisenberg}
\programa{Curso de Bacharelado em Ciências da Computação}
\centro{Departamento de Informática e Estatística}

\def\bancaMembroA{\todo{A definir}}
\def\bancaMembroB{\todo{A definir}}

%----------------------------------------------------------------------
% Preâmbulo
%----------------------------------------------------------------------
\preambulo{Trabalho de Conclusão de Curso submetido ao Curso de Bacharelado em
           Ciências da Computação para a obtenção do Grau de Bacharel
           em Ciências da Computação.}
\assuntos{Ciências da Computação,Modelos,Teses,OpenSource,LaTeX}

\renewcommand{\imprimircapa}{%
    \begin{capa}%
        \center
        {\ABNTEXchapterfont\large\MakeUppercase{\imprimirinstituicao}\\
        \ABNTEXchapterfont\large\MakeUppercase{\imprimircentro}}

        \vspace*{1cm}

        {{\normalfont\large\imprimirautor}}

        \vspace*{4cm}
        \begin{center}
            \ABNTEXchapterfont\bfseries\large\MakeUppercase{\imprimirtitulo}
        \end{center}
        \vfill

        \large\imprimirlocal\\
        \large\the\year
        \vspace*{1cm}
    \end{capa}
}

%--------------------------------------------------------------------------
% Início do documento

\begin{document}
%--------------------------------------------------------------------------
% Elementos pré-textuais
\pretextual
\imprimircapa

\imprimirfolhaderosto

\afterpage{\null\newpage}

%--------------------------------------------------------------------------
% folha de aprovação de proposta de TCC
\begin{snugshade}
    \begin{center}
        {\textbf{\small{FOLHA DE APROVAÇÃO DE PROPOSTA DE TCC}}}
    \end{center}
\end{snugshade}

\vspace{-8pt}
\noindent\resizebox{\textwidth}{!}{
    \footnotesize
    \begin{tabular}{|l|X p{8cm}|}
        \hline
        \textbf{Acadêmico} &  \imprimirautor \\ \hline
        \textbf{Título do trabalho} & \imprimirtitulo \\ \hline
        \textbf{Curso} & Ciências da Computação/INE/UFSC \\ \hline
        \textbf{Área de Concentração} &  Inteligência Artificial \\ \hline
    \end{tabular}
}

\vspace{8pt}

{%
    \small
    \noindent\textbf{Instruções para preenchimento pelo \uline{ORIENTADOR DO TRABALHO}:}
    \begin{itemize}[leftmargin=*,noitemsep,topsep=0pt]
        \item[-] Para cada critério avaliado, assinale um X na coluna SIM
            apenas se considerado aprovado. Caso contrário, indique as
            alterações necessárias na coluna Observação.
    \end{itemize}
}

\vspace{8pt}

\noindent\resizebox{\textwidth}{!}{%
    \footnotesize
    \begin{tabular}{|X p{6cm}|X p{0.5cm}|X p{0.8cm}|X p{0.5cm}|X p{1cm}|X p{3.2cm}|}
        \hline
        \shadecell & \multicolumn{4}{c|}{\shadecell \textbf{Aprovado}} & \shadecell \\ \hhline{*{1}{>{\arrayrulecolor{shadecolor}}-}*{4}{>{\arrayrulecolor{black}}|-}>{\arrayrulecolor{shadecolor}}|->{\arrayrulecolor{black}}}
        \multirow{-1}{*}{\shadecell \textbf{Critérios}} & \shadecell\textbf{Sim} &  \shadecell\textbf{Parcial} & \shadecell\textbf{Não} & \shadecell\textbf{Não se aplica} & \multirow{-1}{*}{\shadecell \textbf{Observação}} \\ \hline
        1. O trabalho é adequado para um TCC no CCO/SIN (relevância/abrangência)? & \shadecell  & \shadecell & \shadecell  & \shadecell  & \\ \hline
        2. O título do trabalho é adequado? & \shadecell & \shadecell & \shadecell & \shadecell & \\ \hline
        3. O tema de pesquisa está claramente descrito? & \shadecell & \shadecell & \shadecell  & \shadecell & \\ \hline
        4. O problema/hipóteses de pesquisa do trabalho está claramente identificado? & \shadecell  & \shadecell  & \shadecell & \shadecell & \\ \hline
        5. A relevância da pesquisa é justificada? & \shadecell & \shadecell & \shadecell  & \shadecell & \\ \hline
        6. Os objetivos descrevem completa e claramente o que se pretende alcançar neste trabalho? & \shadecell & \shadecell & \shadecell & \shadecell & \\ \hline
        7. É definido o método a ser adotado no trabalho? O método condiz com os objetivos e é adequado para um TCC? & \shadecell & \shadecell & \shadecell & \shadecell & \\ \hline
        8. Foi definido um cronograma coerente com o método definido (indicando todas as atividades) e com as datas das entregas (p.ex.Projeto I, II, Defesa)? & \shadecell & \shadecell & \shadecell & \shadecell & \\ \hline
        9. Foram identificados custos relativos à execução deste trabalho (se houver)? Haverá financiamento para estes custos? & \shadecell & \shadecell & \shadecell & \shadecell & \\ \hline
        10. Foram identificados todos os envolvidos neste trabalho? & \shadecell & \shadecell & \shadecell & \shadecell & \\ \hline
        11. As formas de comunicação foram definidas (ex: horários para orientação)? & \shadecell  & \shadecell & \shadecell & \shadecell & \\ \hline
        12. Riscos potenciais que podem causar desvios do plano foram identificados? & \shadecell  & \shadecell & \shadecell & \shadecell & \\ \hline
        13. Caso o TCC envolva a produção de um software ou outro tipo de produto e seja desenvolvido também como uma atividade  realizada numa empresa ou laboratório, consta da proposta uma declaração (\anexoname\ \ref{decl-concor}) de ciência e concordância com a entrega do código fonte e/ou documentação produzidos? & \shadecell & \shadecell & \shadecell & \shadecell & \\ \hline
    \end{tabular}
}

\vspace{12pt}


\noindent\resizebox{\textwidth}{!}{
    \scriptsize
    \begin{tabular}{|X p{3cm}|X p{2.35cm}|X p{1.6cm}|X p{3.4cm}|}
        \hline
        \textbf{Avaliação} & \multicolumn{1}{l}{\textbf{$\Box$ \footnotesize Aprovado}} & \multicolumn{2}{c|}{\textbf{$\Box$ \footnotesize Não Aprovado}} \\ \hline \hline
        \textbf{Professor Responsável} &  {\footnotesize \imprimirorientador} & & \\ \hline
        \textbf{Orientador} & {\footnotesize \imprimirorientador} & & \\ \hline
    \end{tabular}
}

\clearpage
\afterpage{\null\newpage}

%--------------------------------------------------------------------------
\begin{resumo}
    % TODO: Resumo

    \vspace{\onelineskip}
    \noindent
    \textbf{Palavras-chave}: \listaassuntos
\end{resumo}

\afterpage{\null\newpage}

\begin{KeepFromToc}
    \tableofcontents
\end{KeepFromToc}

%--------------------------------------------------------------------------
\chapter{Introdução}
A caracterização de reservatórios de petróleo e gás se beneficiou de avanços tecnológicos como a sísmica 3D e atributos sísmicos, que fornecem informações qualitativas da geometria e dos parâmetros físicos da subsuperfície  \cite{Sancevero:2007}. Essas tecnologias deram origem  ao desafio de análise dessa grande quantidade de dados disponíveis aos intérpretes. 

No processo de análise de fácies sísmicas, a abordagem não supervisionada utiliza atributos sísmicos como entrada de um algoritmo de clusterização. Nesse contexto, o algoritmo de redes de Kohonen, ou \textit{Self Organizing Maps}, (SOM) teve melhores resultados \cite{Coleou:2003} e atualmente é utilizado em \textit{softwares} comerciais. Entretanto, resultados úteis dependem mais dos atributos selecionados do que o método de clusterização utilizado \cite{Barnes:2002}. A seleção de um bom conjunto de atributos dentre a centenas disponíveis representa um problema em questão de tamanho e dimensionalidade dos dados. Além disso, nem todo atributo é relevante para a análise e muitos deles contém informações redundantes \cite{Barnes:2007}.

Desse modo, faz-se necessário o desenvolvimento de métodos para selecionar automaticamente um subconjunto de atributos sísmicos relevantes para a classificação de fácies sísmicas. No contexto de previsão de \textit{logs} de poços, que possui o mesmo desafio de seleção de atributos, uma abordagem utilizando algoritmos genéticos e redes neurais apresentou bons resultados \cite{Dorrington:2004}. 

%--------------------------------------------------------------------------
\chapter{Objetivos}

\section{Objetivo Geral}

\section{Objetivos Específicos}


%--------------------------------------------------------------------------
\chapter{Método de Pesquisa}

%--------------------------------------------------------------------------
\chapter{Cronograma}

\noindent\resizebox{\textwidth}{!}{

    \rowcolors{0}{shadecolor}{shadecolor}
    \begin{tabular}{|X p{3cm}|c|c|c|c|c|c|c|c|c|c|c|c|}
        \hline
            \multirow{-1}{*}{\textbf{Etapas}} &
    
            \multicolumn{12}{|c|}{\textbf{Meses}} \\ \hhline{|~|*{12}{-|}}
            & \textbf{Jan.} & \textbf{Fev.} & \textbf{Mar.} & \textbf{Abr.} & \textbf{Mai.} & \textbf{Jun.} & \textbf{Jul.} & \textbf{Ago.} & \textbf{Set.} & \textbf{Out.} & \textbf{Nov.} & \textbf{Dez.} \\ \hline
            \hiderowcolors Atividade 1
            &      &      &      &      &      &      &      &      &      &      &      &      \\ \hline
            \hiderowcolors Atividade 2
            &      &      &      &      &      &      &      &      &      &      &      &      \\ \hline
            \hiderowcolors Atividade 3
            &      &      &      &      &      &      &      &      &      &      &      &      \\ \hline
            \hiderowcolors Atividade 4
            &      &      &      &      &      &      &      &      &      &      &      &      \\ \hline
            \hiderowcolors Atividade 5
            &      &      &      &      &      &      &      &      &      &      &      &      \\ \hline
            \hiderowcolors Atividade 6
            &      &      &      &      &      &      &      &      &      &      &      &      \\ \hline
            \hiderowcolors Atividade 7
            &      &      &      &      &      &      &      &      &      &      &      &      \\ \hline
            \hiderowcolors Atividade 8
            &      &      &      &      &      &      &      &      &      &      &      &      \\ \hline
            \hiderowcolors Atividade 9
            &      &      &      &      &      &      &      &      &      &      &      &      \\ \hline
            \hiderowcolors Entrega do relatório de TCC I
            &      &      &      &      &      &      &      &      &      &      &      &      \\ \hline
            \hiderowcolors Entrega do rascunho da monografia
            &      &      &      &      &      &      &      &      &      &      &      &      \\ \hline
            \hiderowcolors Defesa do TCC
            &      &      &      &      &      &      &      &      &      &      &      &      \\ \hline
        \end{tabular}
}

\chapter{Custos}

\vspace{-.1cm}
\noindent\resizebox{\textwidth}{!}{
\begin{tabular}{|m{7.25cm}|>{\centering}m{1.5cm}|>{\centering}m{1.25cm}|>{\centering}m{1.25cm}|}
    \hline
    {\shadecell} \centering \textbf{Item} & {\shadecell} \textbf{Quantidade} & {\shadecell} \textbf{Valor unitário (R\$)} & {\shadecell} \textbf{Total (R\$)}
    \tabularnewline \hline
    \multicolumn{4}{|l|}{{\shadecell}\textbf{Material permanente}}
    \tabularnewline \hline
    (p.ex. computadores, equipamentos, componentes de hardware, livros, etc.) &    &  &  
    \tabularnewline \hline
    \multicolumn{4}{|l|}{{\shadecell}\textbf{Material de consumo}}
    \tabularnewline \hline
    CD               & 6   & $1,50$ & $9,00$
    \tabularnewline \hline
    Folhas Impressas & 500 & $0,20$ & $100,00$
    \tabularnewline \hline
    \multicolumn{4}{|l|}{{\shadecell}\textbf{Outros recursos e serviços}}
    \tabularnewline \hline
    (p.ex. licenças de uso de software, fotocópias, transporte, viagens, treinamentos, etc.) & & & 
    \tabularnewline \hline
    \textbf{Total} & & & 
    \tabularnewline \hline
\end{tabular}
}

\chapter{Recursos Humanos}

\noindent\resizebox{\textwidth}{!}{
\begin{tabular}{|l|l|}
    \hline
    \rowcolor{shadecolor}
    \multicolumn{1}{|c|}{\textbf{Nome}} &  \multicolumn{1}{|c|}{\textbf{Função}} \\ \hline
    \imprimirautor                   & Autor \\ \hline
    \imprimirorientador              & Orientador \\ \hline
    \imprimirorientador              & Professor Responsável \\ \hline    
    % \bancaMembroA{}                & Membro da Banca \\ \hline
    % \bancaMembroB{}                & Membro da Banca \\ \hline
    Renato Cislaghi                  & Coordenador de Projetos \\ \hline
\end{tabular}
}

\chapter{Comunicação}

\noindent\resizebox{\textwidth}{!}{
    \begin{tabular}{|>{\centering}m{3cm}|>{\centering}m{3cm}|>{\centering}m{3cm}|>{\centering}m{3cm}|>{\centering}m{3cm}|}
        \hhline{|*{5}{-|}}
        {\shadecell} \textbf{O que precisa ser comunicado} & {\shadecell} \textbf{Por quem} & {\shadecell} \textbf{Para quem} & {\shadecell} \textbf{Melhor forma de Comunicação} & {\shadecell} \textbf{Quando e com que frequência} \tabularnewline \hhline{|*{5}{-|}}
        
        Entregas & Autor & Coordenador de Projetos & Site do TCC & Nos dias estipulados pelas disciplinas \tabularnewline \hhline{|*{5}{-|}}
        
        Reuniões com o orientador & Autor & Orientador & Presencial & Quinzenalmente \tabularnewline \hhline{|*{5}{-|}}
        
        Revisões da monografia & Autor & Orientador, membros da banca & Papel impresso ou pdf & Período de elaboração da monografia \tabularnewline \hhline{|*{5}{-|}}
        
        Dúvidas & Autor & Orientador, Membros da Banca ou Coordenador de Projetos & E-mail e/ou presencial & Quando necessário \tabularnewline \hhline{|*{5}{-|}}
    \end{tabular}
}

\chapter{Riscos}

\noindent\resizebox{\textwidth}{!}{
    \begin{tabular}{|m{2cm}|m{1cm}|m{1cm}|m{1cm}|m{3cm}|m{3cm}|}
        \hhline{|*{6}{-|}}
        \multicolumn{1}{|c|}{{\shadecell}\textbf{Risco}} & \multicolumn{1}{|c|}{{\shadecell}\textbf{Probabilidade}} & \multicolumn{1}{|c|}{{\shadecell}\textbf{Impacto}} & \multicolumn{1}{|c|}{{\shadecell}\textbf{Prioridade}} & \multicolumn{1}{|c|}{{\shadecell}\textbf{Estratégia de Resposta}} & \multicolumn{1}{|c|}{{\shadecell}\textbf{Ações Preventivas}} \tabularnewline \hhline{|*{6}{-|}}
        Perda de dados (HD) & Média & Alto & Alta & Recuperação dos dados e aquisição de novo HD. & Versionar desenvolvimento remotamente. Gerar backups periodicamente. \tabularnewline \hhline{|*{6}{-|}}
        Alteração no tema & Baixa & Alto & Alta & Modificar o escopo do tema ou adotar um novo tema. & Manter interação constante com orientador. \tabularnewline \hhline{|*{6}{-|}}
        Alteração no cronograma & Baixa & Alto & Média & Diminuir o escopo do trabalho. & Monitorar continuamente as informações obtidas dos superiores imediatos. \tabularnewline \hhline{|*{6}{-|}}
        Problemas de saúde & Baixa & Média & Média & Realizar tratamento e retomar as atividades o quanto antes. & Ter hábitos saudáveis e ser precavido. \tabularnewline \hhline{|*{6}{-|}}
    \end{tabular}
}

\bibliography{references}

%--------------------------------------------------------
% Elementos pós-textuais

\begin{anexosenv}
    \partanexos
    
    \chapter{DECLARAÇÃO PADRÃO PARA EMPRESA OU LABORATÓRIO}
    \label{decl-concor}

    \begin{snugshade}
        \begin{center}
            \textbf{DECLARAÇÃO DE CONCORDÂNCIA COM AS CONDIÇÕES PARA O DESENVOLVIMENTO DO TCC NA INSTITUIÇÃO}
        \end{center}
    \end{snugshade}

    \vspace{10pt}

    Declaro estar ciente das premissas para a realização de Trabalhos de Conclusão de Curso (TCC) de Ciência da Computação e Sistema de In\-for\-ma\-ções da UFSC, particularmente da necessidade de que se o TCC envolver o desenvolvimento de um software ou produto específico (ex: um protocolo, um método computacional, etc.) o código fonte e/ou documentação completa correspondente deverá ser entregue integralmente, como parte integrante do relatório final do TCC.

    Ciente dessa condição básica, declaro estar de acordo com a realização do TCC identificado pelos dados apresentados a seguir.

    \vspace{20pt}


    \noindent\resizebox{\textwidth}{!}{
        \begin{tabular}{|l|X p{8cm}|}
            \hline
            \textbf{Instituição} & ECL/INE/CTC \\ \hline
            \textbf{Nome do Responsável} & \imprimirorientador \\ \hline
            \textbf{Cargo/Função} & Prof. INE/CTC \\ \hline
            \textbf{Fone de Contato} & (48) 3721 7549\\ \hline
            \textbf{Acadêmico} & \imprimirautor \\ \hline
            \textbf{Título do trabalho} & Verificação pré-silício de memória baseada em testes dirigidos adaptativos\\ \hline
            \textbf{Curso} & Ciências da Computação/INE/UFSC \\ \hline
        \end{tabular}
    }

    \vspace{40pt}

    \begin{flushright}

        Florianópolis, 14 de Dezembro de 2015.

    \end{flushright}

    \vspace{20pt}


    \begin{center}
        \small
        \parbox{7cm}{%
            \centering
            \rule{6cm}{1pt}\\
            \textbf{Professor Responsável}\\
            \imprimirorientador
        }
        \hfill
    \end{center}

\end{anexosenv}

\end{document}

